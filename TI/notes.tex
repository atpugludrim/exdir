% Created 2022-05-12 Thu 14:15
% Intended LaTeX compiler: pdflatex
\documentclass[11pt]{article}
\usepackage[utf8]{inputenc}
\usepackage[T1]{fontenc}
\usepackage[a4paper,margin=1in]{geometry}
\usepackage{graphicx}
\usepackage{grffile}
\usepackage{longtable}
\usepackage{wrapfig}
\usepackage{rotating}
\usepackage[normalem]{ulem}
\usepackage{amsmath}
\usepackage{textcomp}
\usepackage{amssymb}
\usepackage{capt-of}
\usepackage{hyperref}
\date{\today}
\title{Notes for TI}
\hypersetup{
 pdfauthor={},
 pdftitle={Notes for TI},
 pdfkeywords={},
 pdfsubject={},
 pdfcreator={Emacs 27.1 (Org mode 9.3)}, 
 pdflang={English}}
\begin{document}

\maketitle
\tableofcontents


\section{RTOS}
\label{sec:org5f41da6}
\textbf{R}eal \textbf{T}ime \textbf{O}perating \textbf{S}ystem

Key features:
\begin{itemize}
\item \textbf{Minimal} interrupt and thread switching \textbf{latency}.
\item \textbf{Deterministic}
\item \textbf{Scalability}
\item \textbf{Structured Software} (don't know what this one means :()
website says it's easy to add additional components into application
\item \textbf{Offload development} (takes the load off of developer by
doing scheduling, power mangement, memory management, exception
handling, interrupt table management)
\end{itemize}

\textbf{When is an RTOS needed?}

RTOS is used in the context of embedded devices. It is used when
``more'' complicated functionality is needed, more than just a few
simple actions and when scalability is needed. \textbf{Embedded
device/system} is a small computer that forms parts of a larger
system. For example, a computer inside of a microwave to do automatic
turning off, a computer inside a washing machine, air conditioner,
engine systems inside automatic transmission vehicles, TV, phone, GPS
tracker, robotics system (Mars rover), Alexa, etc.
On the other end of RTOS is \textbf{Bare-Metal} applications.

Thus like any CPU, RTOS has:
\begin{itemize}
\item Scheduler
\item Communication mechanism (semaphore, message queues, queues)
\item Critical region mechanism (mutexes, gates, locks)
\item Clocks, timers
\item Power management
\item Memory management (heaps, fixed or variable sized)
\item Peripheral drivers (UART, SPI, I2C)
\item Protocol stacks (WiFi, etc)
\item File system
\item Device management (exception handling, booting, etc)
\end{itemize}

From the components we can see that an RTOS might be required in Alexa
as it needs to connect to WiFi, store commands, understand commands,
connect to other devices get data from them using UART etc. But might
not be required for AC since it just needs to sense the temperature
and turn on or off accordingly, maybe have a timer at max.

\section{UART}
\label{sec:org8902b42}
\textbf{U}niversal \textbf{A}synchronous
\textbf{R}eceiver/\textbf{T}ransmitter

UART does the job of receiving and transmitting data from CPU to
peripheral. It supports the CPU by buffering, by converting serial
data coming from peripherals to parallel for CPU use and converting
parallel data from CPU to serial for transmission to peripherals. It
also stores error information along with data in buffer. The
serial-to-parallel and parallel-to-serial communication is what makes
it \textbf{asynchronous}. It typically functions in a microcomputer system as
a serial input/output interface.

Side question what are peripherals? Mouse, keyboard, monitor, GPS, IMU
(used in robotics, tells speed, rotational speed and heading
direction. Short for Inertial Measurement Unit), seismo sensor,
anything that sense and transmits data.

How this is done? Very complicated hardware design. Not for a CS person.

\section{SPI}
\label{sec:org73e9699}
\textbf{S}erial \textbf{P}eripheral \textbf{I}nterface

Analog devices have digital interface between microcontroller and
peripheral, this is where the SPI is needed. The byte of data sent can
be sent over 8 parallel lines, or one after the other in a single
line. SPI defines the communication protocol for the
latter. Communication requires common understanding of what high and
low voltages are that represent 0s and 1s. Also require common timing
between a master and a slave device; often serial clock from master is
used.

SPI has two control lines: slave select and serial clock. SPI bus can
control multiple slaves, but there can only be one master. Each slave
has a slave select for independent control. Data can be transferred be
full duplex.

Key hardware lines:
\begin{itemize}
\item SS: slave select
\item SCLK: serial clock. syncs data transfer between master and slave.
\item MOSI: master in, slave out. sends data from master to slave.
\item MISO: the opposite. Can be shared between all slave devices.
\end{itemize}

Data might be read during the rising or falling edge of S-Clock. Clock
polarity is also important. These allow for many modes of SPI.

\section{I2C}
\label{sec:org6690b7d}
\textbf{I}nter \textbf{I}ntegrated \textbf{C}ircuit (I\(^2\)C)

Created by Philips Semiconductor in 1982.

Requires only two lines for communication, it has a serial data line
and a serial clock line. Connect to many devices with only two
lines. Simple and economical. Several speed modes.
100 kbps, 400 kbps, 1 Mbps, 3.4 Mbps, 5 Mbps (5 Mbps mode omits some
features and is write-only).

Key hardware lines:
\begin{itemize}
\item SCL - serial clock
\item SDA - serial data line
\item SDA is bi-directional communication, half duplex. Sending
configuration data or conversion data etc.
\item Unique address allows for multiple controllers and multiple targets
on the I2C bus.
\end{itemize}

Hardware implementation is difficult to understand or explain.
\href{https://training.ti.com/sites/default/files/docs/adcs-i2c-introduction-the-protocol-presentation.pdf}{Here}

\section{ARM}
\label{sec:orgfff0496}
\textbf{A}corn \textbf{R}ISC \textbf{M}achine

\textbf{R}educed \textbf{I}nstruction \textbf{S}et \textbf{C}omputing
Protoype chip made in 1985. Low power chip, for 1W, but actuall worked
at 100mW. Apple Mac with M1 has ARM. By 1986 Apple began using ARM for
its R\&D.

\textbf{A}dvanced \textbf{R}ISC \textbf{M}achines (the new ARM) spun
off from it in 1991 from investment from Apple and VLSI. 6 billion
chips per quarter in 2021. Found in smartphones, tablets, laptops,
servers, everywhere.

ARMs ISA (Instruction Set Architecture) is now advanced and mature,
includes cryptography, MTE (Memory Tagging Extension). Now there are
v9 architectures. Called Cortex, A710, X2, Neoverse N2, etc.
Scalable Vector Extension 2 (SVE2) at the heart of world's fastest
supercomputers, better ML, etc.
Arm Confidential Computer Architecture (CCA) (Realms)
run mini hypervisor in chip so banking software can't get access to
hardware, cloud computing, isolation, etc.

Cortex A processors (found in smartphones): 32 bit is dead, 64 bit is
going to come.
4 types of processor.
Cortex M (microcontrollers)
Cortex R (for real time applications)
Neoverse (for servers)

ARM has a simplified, logical and efficient instruction set. Apart
from that, it doesn't seem fruitful to look at the exact instruction
set right now. \href{https://developer.arm.com/documentation/den0013/d/Introduction-to-Assembly-Language/Comparison-with-other-assembly-languages}{Here}.

\section{Compiler tool chain}
\label{sec:org791d967}

Here's what I got on stackoverflow. In order to compile the source
code to run on a target machine (not your machine), you need to
compile it with a specific compiler, linker, debugger, etc. This is
calle cross-compilation and the tools used is called the
toolchain. Say for example you have a toolchain for ARM.

\section{Makefile}
\label{sec:org03bbcee}

\href{https://makefiletutorial.com/}{Here}. Purpose: make the build process easier by writing the commands
to build specific objects in a project as rules.

In simple english, makefile may have rules like
\begin{itemize}
\item For target \emph{object.o} run the following command, make sure its
dependencies exist.
\item For target of type \emph{*.exe} run the following command to build them.
\end{itemize}

It only updates those targets that are needed, not a fresh build every
time, this is decided based on dependencies. It can do much more,
clean out the installation, delete some things, etc. It's a full
language on its own. Makefile can be seen as a script read by the
\texttt{make} or \texttt{cmake} or \texttt{gmake} or other make like utilities.
\end{document}
