% Created 2022-05-12 Thu 21:12
% Intended LaTeX compiler: pdflatex
\documentclass[11pt]{article}
\usepackage[utf8]{inputenc}
\usepackage[T1]{fontenc}
\usepackage[a4paper,margin=1in]{geometry}
\usepackage{graphicx}
\usepackage{grffile}
\usepackage{longtable}
\usepackage{wrapfig}
\usepackage{rotating}
\usepackage[normalem]{ulem}
\usepackage{amsmath}
\usepackage{textcomp}
\usepackage{amssymb}
\usepackage{capt-of}
\usepackage{hyperref}
\newcommand{\qed}{\mbox{}\hspace*{\fill}\nolinebreak\mbox{$\rule{0.6em}{0.6em}$}}
\date{\today}
\title{What is it like to go somewhere for all of eternity?}
\hypersetup{
 pdfauthor={},
 pdftitle={What is it like to go somewhere for all of eternity?},
 pdfkeywords={},
 pdfsubject={},
 pdfcreator={Emacs 27.1 (Org mode 9.3)}, 
 pdflang={English}}
\begin{document}

\maketitle
\section{A problem}
Say you did some bad things in life and go to hell. The devil there devices a
punishment for you, asking you to walk from a mountain to a river where you can
drink water. But the real punishment is that when you are about to reach the
river, when only \(10\%\) of the total distance is left the devil makes your
feet 10\(\times\) slower. And this process keeps repeating, so that when you are
just \(10\%\) of remaining distance away from the river, you are slowed
10\(\times\) from the current speed. Will you ever reach the river?
\par
Let's say distance from mountain to river is \(d\). Speed at start is \(v\),
time taken to reach \(90\%\) of the distance is \(t\). So, when you reach
\(0.9d\), speed becomes \(\frac{v}{10}\). In next \(t\) time you travel
\(\frac{v\times t}{10}=\frac{0.9d}{10}\) which is \(90\%\) of the rest (\(90\%
\text{ of }d(1-0.9)=d(0.1)=d(0.09)\)). In fact, after \(nt\) time, you travel
\(\frac{0.9d}{10^{n-1}}=\frac{9d}{10^n}\) more of the distance towards the river.
\[\text{total distance travelled, }t_n=\sum^n_{i=1}\frac{9d}{10^i}\]

\section{Well what now?}
What is \(\displaystyle\lim_{n\rightarrow\infty}t_n\)? Well let's define \(d=1\)
and define the sequence \(t_n, n\in\{1,2,\dotsc\}\) alternatively. We can see
that the \(t_n\) can be written as \(0.999\dotsm \text{n times}\). Let's define
\(r_n=1-t_n=\frac{1}{10^n}\). Let's define \(m=\frac{t_n+1}{2}\) and notice that
\(t_n\le m\le 1\).
\begin{align*}
    t_n&\le m\le 1\\
    \Rightarrow -t_n &\ge -m \ge -1 \\
    \Rightarrow 1-t_n&\ge 1-m \ge 0\\
    \Rightarrow r_n&\ge 1-m \ge 0 \\
    \Rightarrow \frac{1}{10^n}&\ge 1-m\ge 0
\end{align*}
As \(n\rightarrow\infty\), \(1-m\) gets sandwiched between 0 and 0. Thus at the
limit, \(1-m=0\) which means \(m=1\) which means \(\frac{t_n+1}{2}=1\Rightarrow
t_n+1=2\Rightarrow t_n=1\).\qed\par
Which means, once the eternity is over you'll be able to reach the river, drink
the water and start a new life at the birth of the new universe.

\section{Alternate proof?}
Given a positive number \(\varepsilon >0\), we want to find an integer
\(N_0(\varepsilon)\) such that \hbox{\(\forall\; n\ge N_0(\varepsilon), \lvert
t_n-1\rvert <\varepsilon\)}.
\begin{align*}
    &\lvert t_{N_0}-1\rvert < \varepsilon\\
    &\lvert 1-r_{N_0}-1\rvert < \varepsilon\\
    &\lvert r_{N_0}\rvert < \varepsilon\\
    &\left\lvert \frac{1}{10^{N_0}} \right\rvert< \varepsilon\\
    &\frac{1}{10^{N_0}} < \varepsilon\quad\text{\small since it's always
    positive}\\
    &10^{N_0}>\frac{1}{\varepsilon}\\
    &\Rightarrow N_0 > \log_{10}\left(\frac{1}{\varepsilon}\right)\quad\text{\small because log is an
    increasing function}
\end{align*}
Thus choosing \(N_0=\max\{0,\lceil\log_{10}\frac{1}{\varepsilon}\rceil+1\}\) does the
job.\qed\par
This says that we can get arbitrarily close (\(\varepsilon>0\)) to 1 and there
will be an infinitely many points of the sequence within that
\(\varepsilon\)-ball. This is the definition of convergence of a sequence to a
point, or the definition of existence of a limit. This proves that
\(0.\bar{9}=1\).

\section{What about geometric progression?}
\(S_n=\frac{a(1-r^n)}{1-r}\). Identifying \(a=\frac{9}{10}\) and
\(r=\frac{1}{10}\), \(\displaystyle S_n=\frac{9}{10}\frac{1-\frac{1}{10^n}}{1-\frac{1}{10}}\)
and
\(\displaystyle\lim_{n\rightarrow\infty}S_n=\frac{\frac{9}{10}}{1-\frac{1}{10}}=1\).

\section{Why the hell did I make this?}
Because in school we did a math problem where we proved that \(0.\bar{9}=1\) but
it was wrong, because it can not be algebraically multiplied out. Instead, it's a
``limit'' of a sequence. Which is why in all my previous ways of getting to 1, I
use limits. Just saying \(0.\bar{9}=x\), \(9.\bar{9}=10x\), \(9=9x\), \(x=1\) is
wrong. The addition, multiplication, subtraction and division used here are not
defined. High school math is much more difficult than we think. \par
This is the sin you committed because of which you in hell. You're a very bad
man like Jerry.\par
And this is not
the end of the story, what does the norm \(\lvert\cdot\rvert\) mean. How do we
know that at the limit, the sum of a GP is actually \(\frac{a}{1-r}\) for
\(r<1\). We need to define metrics, look at convergence, Cauchy sequences and
much more to complete the proof.

\end{document}
